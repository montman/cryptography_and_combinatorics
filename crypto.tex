\title{Crittografia e Combinatoria}
\author{Amati Pierluigi}
\date{\today}

\documentclass[12pt]{article}
\usepackage[italian]{babel}
\usepackage{tikz}
\usepackage{amsfonts}
\usetikzlibrary{positioning}
\begin{document}
\maketitle
\tableofcontents

\section{Introduzione}
La \textbf{crittologia} è lo studio dei metodi per mantenere sicure le comunicazioni che avvengono in un canale \textit{non sicuro}. \\
Essa si suddivide in \textbf{crittografia}, che studia la progettazione di tali metodi, e \textbf{crittanalisi}, che invece si occupa di infrangerli.
\begin{center}
\begin{tikzpicture}
\matrix [column sep=7mm, row sep=5mm] {
&
  \node (yw) [draw, shape=rectangle] {Crittologia}; \\
  \node (we) [draw, shape=rectangle] {Crittografia}; & &
  \node (pu) [draw, shape=rectangle] {Crittanalisi}; \\
};
\draw[->, thick] (yw) -- (we);
\draw[->, thick] (yw) -- (pu);
\end{tikzpicture}
\end{center}
In generale in una comunicazione un messaggio \textbf{m} viene cifrato all'origine attraverso un algoritmo di cifratura \textbf{\(ENC\)} e una chiave di cifratura \textbf{\textit{k}} e viene decifrato alla destinazione con un algoritmo di decifratura \textbf{\(DEC\)} (idealmente \(ENC^{-1}\)) e una chiave di decifratura \textbf{\textit{k}}. L'algoritmo di cifratura è solitamente noto a tutte le parti, ma è la chiave ad essere segreta.
\[ENC(m,k)=c\]
\[DEC(c,k)=m\]
Ipotizzando una comunicazione tra Alice e Bob, dove entrambi possiedono le chiavi di cifratura, un soggetto esterno malintenzionato, Eve (a.k.a. Evil), potrebbe:
\begin{itemize}
\item leggere il messaggio;
\item trovare la chiave e quindi decifrare tutti i messaggi scambiati tra Alice e Bob;
\item alterare un messaggio in modo tale da far sembrare che sia effettivamente spedito da una delle due parti;
\item fingersi una delle due parti.
\end{itemize}
\paragraph{Tipologie di cifratura}
Esistono principalmente due tipologie di cifratura:
\begin{itemize}
\item la cifratura \textbf{simmetrica}, in cui la chiave di cifratura è identica alla chiave di decifratura;
\item la cifratura \textbf{asimmetrica}, in cui la chiave di cifratura (generalmente pubblica) è differente dalla chiave di decifratura (privata).
\end{itemize}
\section{Teoria dei numeri}
\subsection{Divisibilità}
\paragraph{Definizione}
\textit{Siano $a,b\in \mathbb{N}$, con $a\neq 0$, si dice che a divide b $(a|b)$ se esiste $k\in \mathbb{N}$} tale che $b=ak$. In altre parole, b è un multiplo di a.
\paragraph{Proprietà} (dimostrazioni\footnote{[Trappe p.64]})
\begin{itemize}
\item $a|a$
\item $a|0$
\item $1|b$
\item se $a|b$ e $b|c$, allora $a|c$
\item se $a|b$ e $a|c$, allora $a|(sb + tc)$, con $s,t\in \mathbb{N}$
\end{itemize}
\subsection{Teorema dei numeri primi}
\textit{Sia $\Pi (x)$ la quantità di numeri primi $<x$,} definita $\Pi (x) \simeq \frac{x}{\ln{(x)}}$,
$$\Rightarrow \lim_{x\to\infty} \frac{\Pi (x)\ln{(x)}}{x}.$$
\subsection{Teorema fondamentale dell'aritmetica}
\paragraph{Enunciato} \textit{Ogni numero $n\in \mathbb{N}$ è un prodotto di numeri primi.}
$$n=p_1^{\alpha _1}p_2^{\alpha _2}\cdots p_s^{\alpha _s}$$
Si dice \textbf{fattorizzazione} la ricerca di tale insieme di numeri primi.
\paragraph{Definizione} \textit{Si dice Massimo Comun Divisore tra a e b, il più grande numero intero che divide a e b $[gcd(a,b)]$.}
\paragraph{Algoritmo di Eulero}
\paragraph{Identità di Bezout}



\end{document}
  