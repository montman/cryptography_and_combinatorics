\title{Crittografia e Combinatoria}
\author{Amati Pierluigi}
\date{\today}

\documentclass[a4paper]{report}
\usepackage[italian]{babel}
\usepackage{tikz}
\usepackage{amsfonts}
\usepackage{amsmath}
\usepackage{amssymb}
\usetikzlibrary{positioning}
\usetikzlibrary{arrows}
\begin{document}
\maketitle
\tableofcontents

\chapter{Introduzione}
La \textbf{crittologia} è lo studio dei metodi per mantenere sicure le comunicazioni che avvengono in un canale \textit{non sicuro}.

Essa si suddivide in \textbf{crittografia}, che studia la progettazione di tali metodi, e \textbf{crittanalisi}, che invece si occupa di infrangerli.
\begin{center}
\begin{tikzpicture}
\matrix [column sep=7mm, row sep=5mm] {
&
  \node (yw) [draw, shape=rectangle] {Crittologia}; \\
  \node (we) [draw, shape=rectangle] {Crittografia}; & &
  \node (pu) [draw, shape=rectangle] {Crittanalisi}; \\
};
\draw[->, thick] (yw) -- (we);
\draw[->, thick] (yw) -- (pu);
\end{tikzpicture}
\end{center}
In generale in una comunicazione un messaggio \textbf{m} viene cifrato all'origine attraverso un algoritmo di cifratura \textbf{\(ENC\)} e una chiave di cifratura \textbf{\textit{k}} e viene decifrato alla destinazione con un algoritmo di decifratura \textbf{\(DEC\)} (idealmente \(ENC^{-1}\)) e una chiave di decifratura \textbf{\textit{k}}. L'algoritmo di cifratura è solitamente noto a tutte le parti, ma è la chiave ad essere segreta.
$$ENC(m,k)=c$$
$$DEC(c,k)=m$$
\tikzstyle{int}=[draw, fill=white!20, minimum size=2em]
\tikzstyle{init} = [pin edge={to-,thin,black}]
\tikzstyle{branch}=[fill,shape=circle,minimum size=3pt,inner sep=0pt]
\begin{center}
\begin{tikzpicture}[node distance=3cm,auto,>=latex']
    \node [int,pin={[init]above:$k$}] (a) {$ENC(k,m)$};
    \node [int] (b) [left of=a, node distance=3cm]{Alice};
    \node [int,pin={[init]above:$k$}] (c) [right of=a] {$DEC(k,c)$};
    \node [int] (end) [right of=c, node distance=3cm]{Bob};
    \draw[->] (b) edge node {$m$} (a);
    \draw[->] (a) edge node [name=crypt] {$c$} (c);
    \draw[->] (c) edge node {$m$} (end) ;
    \node [int] (e) [below of=crypt, node distance=2cm]{Eve};
    \draw [->] (crypt) -- (e);
\end{tikzpicture}
\end{center}

Ipotizzando una comunicazione tra Alice e Bob, dove entrambi possiedono le chiavi di cifratura, un soggetto esterno malintenzionato, Eve (a.k.a. Evil), potrebbe:
\begin{itemize}
\item leggere il messaggio;
\item trovare la chiave e quindi decifrare tutti i messaggi scambiati tra Alice e Bob;
\item alterare un messaggio in modo tale da far sembrare che sia effettivamente spedito da una delle due parti;
\item fingersi una delle due parti.
\end{itemize}
\paragraph{Tipologie di cifratura}
Esistono principalmente due tipologie di cifratura:
\begin{itemize}
\item la cifratura \textbf{simmetrica}, in cui la chiave di cifratura è identica alla chiave di decifratura;
\item la cifratura \textbf{asimmetrica}, in cui la chiave di cifratura (generalmente pubblica) è differente dalla chiave di decifratura (privata).
\end{itemize}
\chapter{Teoria dei numeri}
\section{Divisibilità}
\paragraph{Definizione}
\textit{Siano $a,b\in \mathbb{Z}$, con $a\neq 0$, si dice che a divide b $(a\mid b)$ se esiste $k\in \mathbb{Z}$} tale che $b=ak$. In altre parole, b è un multiplo di a.
\paragraph{Proprietà} (dimostrazioni\footnote{[Trappe p.64]})
\begin{itemize}
\item $a\mid a$;
\item $a\mid 0$;
\item $1\mid b$;
\item se $a\mid b$ e $b\mid c$, allora $a\mid c$;
\item se $a\mid b$ e $a\mid c$, allora $a\mid (sb + tc)$, con $s,t\in \mathbb{Z}$.
\end{itemize}
\paragraph{Esempi}
$$15\mid 60,\quad 2\mid 8,\quad 4\nmid 15$$
\section{Teorema dei numeri primi}
\textit{Sia $\Pi (x)$ la quantità di numeri primi $<x$,} definita $\Pi (x) \simeq \frac{x}{\ln{(x)}}$,
$$\Rightarrow \lim_{x\to\infty} \frac{\Pi (x)\ln{(x)}}{x}=1.$$
\section{Teorema fondamentale dell'aritmetica}
\paragraph{Enunciato} \textit{Ogni numero $n\in \mathbb{Z}$ è un prodotto di numeri primi.}
$$n=p_1^{\alpha _1}p_2^{\alpha _2}\cdots p_s^{\alpha _s}$$
Si dice \textbf{fattorizzazione} la ricerca di tale insieme di numeri primi.
\paragraph{Definizione} \textit{Si dice Massimo Comun Divisore tra a e b, il più grande numero intero che divide a e b $[gcd(a,b)]$.}
\paragraph{Definizione} \textit{Dati $a,b\in \mathbb{Z}$, essi si dicono \textbf{co-primi} se e solo se $gcd(a,b)=1$.}
\subsection{Algoritmo Euclideo} Oltre al classico metodo di scomposizione in fattori primi, la ricerca del massimo comun divisore tra due numeri interi è possibile attraverso il cosiddetto Algoritmo Euclideo, di natura iterativa.

Supponendo di voler calcolare $gcd(r_0,r_1)$, con $r_0>r_1$, si può scrivere iterativamente:
$$r_0=r_1\cdot q_1 + r_2$$
$$r_1=r_2\cdot q_2 + r_3$$
$$r_2=r_3\cdot q_3 + r_4$$
$$\vdots$$
dove $r_n$ e $q_n$ sono rispettivamente il resto e il quoziente della divisione tra $r_{n-1}$ e $r_{n-2}$.
$$gcd(r_0,r_1)=r_n \Leftrightarrow r_{n+1}=0$$
\paragraph{Esempio}
Si calcoli $gcd(1180,482)$:

$$1180=482\cdot 2 + 216$$
$$482=216\cdot 2 + 50$$
$$216=50\cdot 4 + 16$$
$$50=16\cdot 3 + 2$$
$$16 = 2\cdot 8 + 0$$
Il resto appena precedente a $r=0$ è $r=2$, quindi $gcd(1180,482)=2$.
\subsection{Identità di Bezout}
\paragraph{Enunciato} \textit{Siano $a,b \in \mathbb{Z}$, dove almeno uno tra a e b è diverso da 0 $\Rightarrow \exists x,y \in \mathbb{Z}: ax+by=gcd(a,b)$.}

Per individuare i valori $x,y$ che soddisfano l'identità di Bezout, si scrivono le espressioni dei resti a partire dalle iterazioni dell'algoritmo Euclideo
$$r_n=r_{n-2}-r_{n-1}\cdot q_{n-1}$$
e si sostituiscono a ritroso, svolgendo i calcoli, fino ad arrivare alla prima iterazione. In riferimento al $gcd(1180,482)$, bisogna trovare i valori $x,y: 1180x+482y=gcd(1180,482)=2$:

$$216=1180-482\cdot 2$$
$$50=482-216\cdot 2$$
$$16=216-50\cdot 4$$
$$2=50-16\cdot 3$$
Si parte quindi dall'ultima identità $2=50-16\cdot 3$, con l'obiettivo di ottenere nel membro di destra un'espressione del tipo $1180x+482y$, sostituendo i valori delle costanti note dalle iterazioni precedenti:

$$2=50-3\cdot(216-50\cdot 4)=50-3\cdot 216 + 12\cdot 50=13\cdot 50 - 3\cdot 216$$
$$2=13\cdot (482-216\cdot2) - 3\cdot 216=13\cdot 482 -26\cdot 216-3\cdot 216=13\cdot 482 -29\cdot 216$$
$$2=13\cdot 482 - 29\cdot (1180-482\cdot 2)=13\cdot 482-29\cdot 1180 + 58\cdot 482$$
$$\Rightarrow - 1180\cdot 29 + 482\cdot 71 = 2 $$
$$\Rightarrow (x,y)=(-29,71)$$

In alternativa è possibile utilizzare il seguente algoritmo iterativo, dove $q_n$ è l'n-esimo quoziente dell'algoritmo Euclideo con $r_{n+1}\neq 0$:

$$y_0=0$$
$$y_1=1$$
$$y_n=-q_{n-1}\cdot y_{n-1}+y_{n-1}$$

da cui si ottiene, nell'esercizio in esame:

$$y_0=0$$
$$y_1=1$$
$$y_2=-2$$
$$y_3=-2\cdot (-2)+1=5$$
$$y_4=-4\cdot5-2=-22$$
$$y_5=-3\cdot (-22)+5=71$$
$$\Rightarrow y=71$$

Si ha quindi $1180\cdot x + 482\cdot 71 = 2$, da cui si ricava $x=-29$.

\paragraph{Esercizio} Individuare $x,y:1234x+1111y=gcd(1234,1111)$.
\chapter{Aritmetica Modulare}
\section{Congruenze}
\paragraph{Teorema}
\textit{Siano $a,b\in \mathbb{Z}, N\in \mathbb{N}\setminus \{ 0\} $, si dice che $a\equiv b \mod{N} \Leftrightarrow \exists k\in \mathbb{Z}:a-b=kN$, cioè se e solo se $a-b$ è un multiplo di N.}
\paragraph{Proprietà} Siano $a,b,c\in \mathbb{Z}, N\in \mathbb{N}\setminus \{ 0\} $:

\begin{itemize}
\item $a\equiv 0 \mod{N} \Leftrightarrow N\mid a$, cioè se $a=kN$;
\item $a\equiv a\mod{N}$, cioè per $k=0$;
\item $a\equiv b\mod{N} \Leftrightarrow b\equiv a\mod{N}$, cioè per $k_1=-k_2$;
\item $a\equiv b\mod {N}, b\equiv c\mod{N} \Rightarrow a\equiv c\mod {N}$, cioè per $k_3=k_1-k_2$.
\end{itemize}

In altre parole, sia $a\in \mathbb{Z}$ un numero divisibile per $N$, esso si può scrivere come:
$$a=kN+r$$
con $k\in \mathbb{Z}$, dove $0\leq r < N$ è il resto della divisione per $N$. Si dice quindi che due numeri $a,b$ sono congruenti se e solo se hanno lo stesso resto se divisi per $N$.
$$a=q_1\cdot N + r$$
$$b=q_2\cdot N + r$$
$$a-b=(q_1-q_2)\cdot N$$
$$a-b=kN\Rightarrow a\equiv b\equiv r \mod{N}$$
\paragraph{Definizione} \textit{Si denota con $\mathbb{Z}_N$ l'insieme di tutti i possibili resti in una divisione per $N$}
$$\mathbb{Z}_N=\{0,1,2,\cdots,N-1\}$$
\paragraph{Lemma} \textit{Siano $a,b,c,d\in \mathbb{Z}, N\in \mathbb{N}\setminus \{ 0\} : a\equiv b, c\equiv d \mod{N}$}
$$\Rightarrow a+c\equiv b+d,\quad a-c\equiv b-d,\quad a\cdot c\equiv b\cdot d \mod{N}$$
per dimostrarlo si utilizzi la definizione di congruenza $a=b+kN$.

\textit{Vale inoltre, dalla regola della moltiplicazione $a\cdot c\equiv b\cdot d \mod{N}$,}
$$a^k\equiv b^k \mod{N}.$$
\section{Divisione in $\mathbb{Z}_N$}
\paragraph{Proposizione} 
\textit{Siano $a,b,c\in \mathbb{Z}, N\in \mathbb{N}\setminus \{ 0\},\quad a\neq 0 $}:
\begin{itemize}
\item $ab\equiv ac \mod{aN} \Rightarrow b\equiv c\mod{N}$, poiché se
$$ab\equiv ac \mod{aN} \Rightarrow \exists k\in \mathbb{Z}: ab=kaN+ac\Rightarrow b=kN+c$$
$$\Rightarrow b\equiv c \mod{N}$$
\item $ab\equiv ac \mod{N}, gcd(a,N)=1 \Rightarrow b\equiv c\mod{N}$, poiché se
$$gcd(a,N)=1 \Rightarrow \exists x,y\in \mathbb{Z}: ax+Ny=1,$$
si moltiplichi l'equazione per $(b-c)$, ottenendo:
$$(ab-ac)y+n(b-c)y=b-c$$
dato che per ipotesi $(ab-ac)$ è un multiplo di $N$, e osservando che $N(b-c)y$ è un multiplo di $N$, si deduce che anche $b-c$ è un multiplo di $N$.
$$\Rightarrow b\equiv c \mod{N}.$$
\end{itemize}
\section{Gruppi}
\paragraph{Definizione} \textit{Definito un insieme $\mathbb{G} \neq \{\emptyset\}$ e un'operazione $*$ tale che $\forall a,b \in \mathbb{G}\Rightarrow a*b\in \mathbb{G}$, la struttura $(\mathbb{G},*)$ si dice \textbf{gruppo} se:}
\begin{itemize}
\item $\exists e \in \mathbb{G}: a*e=e*a=a,\quad \forall a \in \mathbb{G}$;
\item $\exists b \in \mathbb{G}: a*b=b*a=e,\quad \forall a \in \mathbb{G}$;
\item $(a*b)*c = a*(b*c), \quad \forall a,b,c \in \mathbb{G}$.
\end{itemize}
\paragraph{Definizione} \textit{Un gruppo $(\mathbb{G},*)$, si dice \textbf{Abeliano}, o \textbf{commutativo} se:}
\begin{itemize}
\item $a*b=b*a, \quad \forall a,b\in \mathbb{G}$. 
\end{itemize}
\paragraph{Lemma}\textit{$(\mathbb{Z}_N,+)$ è un gruppo Abeliano.}
\paragraph{Lemma}\textit{$(\mathbb{Z}_N\setminus \{ 0\},\cdot )$ è un gruppo $\Leftrightarrow N$ è primo. }
\paragraph{Esempi} Per verificare se una struttura $(\mathbb{G},*)$ sia un gruppo, è sufficiente applicare l'operazione considerata a tutte le possibili combinazioni degli elementi appartenenti all'insieme $\mathbb{G}$.

N.B. Nel caso di insiemi $\mathbb{Z}_N$ tutte le operazioni vanno effettuate in $\mod{N}$.
\begin{center}
\begin{tabular}{ c | c c c c c c}
 + & 0 & 1 & 2 & 3 & 4 & 5 \\ 
 \hline
 0 & 0 & 1 & 2 & 3 & 4 & 5 \\
 1 & 1 & 2 & 3 & 4 & 5 & 0 \\
 2 & 2 & 3 & 4 & 5 & 0 & 1 \\
 3 & 3 & 4 & 5 & 0 & 1 & 2 \\
 4 & 4 & 5 & 0 & 1 & 2 & 3 \\
 5 & 5 & 0 & 1 & 2 & 3 & 4    
\end{tabular}
\end{center}
Nella tabella si è scelto $\mathbb{Z}_6$ con l'operazione di addizione $\mod{6}$, con $e=0$. Per ogni $a$ appartenente all'insieme, esiste $b$ tale che $a+b=b+a=e=0$, e inoltre $a+e=e+a=a$.
\begin{center}
\textit{$\Rightarrow (\mathbb{Z}_6,+)$ è un gruppo.}
\end{center}
\begin{center}
\begin{tabular}{ c | c c c }
$\cdot$ & 1 & 2 & 3\\ 
 \hline
 1 & 1 & 2 & 3 \\
 2 & 2 & 1 & 3 \\
 3 & 3 & 3 & 3 
\end{tabular}
\end{center}
In questo caso si è scelta la struttura $(\mathbb{Z}_4\setminus \{ 0\},\cdot )$, dove $N=4$ non è primo. Il valore $e$ tale che $a\cdot e = e\cdot a = a$ è $1$. La seconda proprietà invece non è rispettata, poiché non esiste nessun numero appartenente all'insieme che moltiplicato per $3$, dà come risultato $e=1$.
\begin{center}
\textit{$\Rightarrow (\mathbb{Z}_4\setminus \{ 0\},\cdot )$ \textbf{non} è un gruppo.}
\end{center}
\end{document}
  